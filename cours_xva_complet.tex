\documentclass[11pt,a4paper]{article}

% ============================================================================
% PACKAGES
% ============================================================================
\usepackage[utf8]{inputenc}
\usepackage[T1]{fontenc}
\usepackage[french]{babel}
\usepackage[margin=2.5cm]{geometry}
\usepackage{amsmath,amssymb}
\usepackage{graphicx}
\usepackage{xcolor}
\usepackage{tikz}
\usetikzlibrary{shapes,arrows,positioning,calc}
\usepackage{tcolorbox}
\usepackage{booktabs}
\usepackage{enumitem}
\usepackage{fancyhdr}
\usepackage{hyperref}

% ============================================================================
% COULEURS
% ============================================================================
\definecolor{darkblue}{RGB}{0, 51, 102}
\definecolor{lightblue}{RGB}{230, 242, 255}
\definecolor{darkgreen}{RGB}{0, 100, 0}
\definecolor{lightgreen}{RGB}{230, 255, 230}
\definecolor{darkred}{RGB}{150, 0, 0}
\definecolor{lightred}{RGB}{255, 230, 230}
\definecolor{darkorange}{RGB}{200, 100, 0}
\definecolor{lightorange}{RGB}{255, 245, 230}
\definecolor{darkpurple}{RGB}{75, 0, 130}
\definecolor{lightpurple}{RGB}{245, 230, 255}

% ============================================================================
% BOÎTES PERSONNALISÉES
% ============================================================================
\tcbset{
    boxrule=1pt,
    arc=3mm,
    fonttitle=\bfseries
}

\newtcolorbox{definition}[1]{
    colback=lightblue,
    colframe=darkblue,
    title=#1
}

\newtcolorbox{exemple}[1]{
    colback=lightgreen,
    colframe=darkgreen,
    title=Exemple : #1
}

\newtcolorbox{attention}[1]{
    colback=lightred,
    colframe=darkred,
    title=#1
}

\newtcolorbox{formule}[1]{
    colback=lightorange,
    colframe=darkorange,
    title=Formule : #1
}

\newtcolorbox{resume}[1]{
    colback=lightpurple,
    colframe=darkpurple,
    title=#1
}

% ============================================================================
% EN-TÊTE
% ============================================================================
\pagestyle{fancy}
\fancyhf{}
\fancyhead[L]{\textcolor{darkblue}{Cours xVA - De Zéro à la Compréhension}}
\fancyhead[R]{\textcolor{darkblue}{\thepage}}
\renewcommand{\headrulewidth}{0.5pt}

% ============================================================================
% DOCUMENT
% ============================================================================
\begin{document}

% --------------------------------------------------------------------------
% PAGE DE TITRE
% --------------------------------------------------------------------------
\begin{titlepage}
\centering
\vspace*{2cm}

{\Huge\bfseries\textcolor{darkblue}{Comprendre les xVA}}\\[0.5cm]
{\LARGE\textcolor{darkblue}{De Zéro à la Maîtrise}}\\[1cm]

{\Large Cours Complet avec Exemples Simples}\\[2cm]

\begin{tikzpicture}
\node[draw, rounded corners, fill=lightblue, minimum width=12cm, minimum height=3cm, align=center] {
    {\Large\textbf{Ce que tu vas apprendre :}}\\[0.3cm]
    Produits dérivés $\bullet$ Risque de contrepartie $\bullet$ Exposition\\
    Collatéral $\bullet$ Monte Carlo $\bullet$ CVA, DVA, FVA, MVA, KVA $\bullet$ SA-CCR
};
\end{tikzpicture}

\vfill

{\large Préparation à la présentation de l'application xVA Calculator}\\[1cm]
{\large\today}

\end{titlepage}

% --------------------------------------------------------------------------
% TABLE DES MATIÈRES
% --------------------------------------------------------------------------
\tableofcontents
\newpage

% ============================================================================
% CHAPITRE 1 : LES BASES
% ============================================================================
\section{Les Bases : C'est quoi un Produit Dérivé ?}

\subsection{Définition Simple}

\begin{definition}{Produit Dérivé}
Un \textbf{produit dérivé} est un contrat financier dont la valeur \textbf{dépend} (dérive) d'un autre actif appelé "sous-jacent".

\vspace{0.3cm}
\textbf{Exemples de sous-jacents :}
\begin{itemize}
    \item Taux d'intérêt (ex: Euribor, OIS)
    \item Taux de change (ex: EUR/USD)
    \item Actions (ex: CAC40)
    \item Matières premières (ex: pétrole)
\end{itemize}
\end{definition}

\subsection{Les Deux Produits de Notre Application}

Notre application gère deux types de produits dérivés :

\subsubsection{1. Le Swap de Taux d'Intérêt (IRS)}

\begin{definition}{Interest Rate Swap (IRS)}
Un \textbf{swap de taux} est un accord entre deux parties pour \textbf{échanger des flux d'intérêts} pendant une certaine durée.

\begin{itemize}
    \item \textbf{Une partie paie un taux fixe} (ex: 2\% par an)
    \item \textbf{L'autre partie paie un taux variable} (ex: Euribor)
\end{itemize}
\end{definition}

\begin{exemple}{Swap de Taux}
\textbf{Situation :} Banque A et Entreprise B signent un swap sur 10 millions d'euros pendant 5 ans.

\begin{center}
\begin{tikzpicture}[scale=0.9]
    \node[draw, rounded corners, fill=lightblue, minimum width=3cm, minimum height=1.5cm] (A) at (0,0) {\textbf{Banque A}};
    \node[draw, rounded corners, fill=lightgreen, minimum width=3cm, minimum height=1.5cm] (B) at (8,0) {\textbf{Entreprise B}};

    \draw[->, very thick, darkred] (A.east) -- node[above] {Paie taux fixe 2\%} (B.west);
    \draw[->, very thick, darkgreen] (B.west) ++(0,-0.3) -- node[below] {Paie taux variable (Euribor)} ($(A.east)+(0,-0.3)$);
\end{tikzpicture}
\end{center}

\textbf{Chaque année :}
\begin{itemize}
    \item Banque A paie : $10\,000\,000 \times 2\% = 200\,000$ EUR (fixe)
    \item Entreprise B paie : $10\,000\,000 \times \text{Euribor}$ (variable)
\end{itemize}

\textbf{Si Euribor = 1.5\% :} Banque A paie 200k, reçoit 150k $\Rightarrow$ perd 50k\\
\textbf{Si Euribor = 3\% :} Banque A paie 200k, reçoit 300k $\Rightarrow$ gagne 100k
\end{exemple}

\subsubsection{2. Le Forward de Change (FX Forward)}

\begin{definition}{FX Forward}
Un \textbf{forward de change} est un accord pour \textbf{échanger des devises à une date future} à un taux fixé aujourd'hui.
\end{definition}

\begin{exemple}{FX Forward}
\textbf{Situation :} Aujourd'hui, le taux EUR/USD est 1.10 (1 EUR = 1.10 USD).

Une entreprise française signe un forward pour acheter 1 million d'EUR dans 1 an au taux de 1.12.

\begin{center}
\begin{tikzpicture}
    \node[draw, rounded corners, fill=lightblue, minimum width=10cm, minimum height=1.2cm, align=center] at (0,0) {
        Dans 1 an : L'entreprise paiera \textbf{1,120,000 USD} pour recevoir \textbf{1,000,000 EUR}
    };
\end{tikzpicture}
\end{center}

\textbf{Dans 1 an, si EUR/USD = 1.20 :}
\begin{itemize}
    \item Prix marché : 1,200,000 USD pour 1M EUR
    \item Prix forward : 1,120,000 USD pour 1M EUR
    \item \textbf{Gain : 80,000 USD} (on achète moins cher que le marché)
\end{itemize}

\textbf{Dans 1 an, si EUR/USD = 1.05 :}
\begin{itemize}
    \item Prix marché : 1,050,000 USD pour 1M EUR
    \item Prix forward : 1,120,000 USD pour 1M EUR
    \item \textbf{Perte : 70,000 USD} (on achète plus cher que le marché)
\end{itemize}
\end{exemple}

\newpage
% ============================================================================
% CHAPITRE 2 : RISQUE DE CONTREPARTIE
% ============================================================================
\section{Le Risque de Contrepartie}

\subsection{C'est quoi le Risque de Contrepartie ?}

\begin{definition}{Risque de Contrepartie}
Le \textbf{risque de contrepartie} est le risque que l'autre partie du contrat \textbf{fasse défaut} (faillite) et ne puisse pas honorer ses obligations.
\end{definition}

\begin{exemple}{Illustration du Risque}
Tu as un swap avec la banque XYZ. Le swap vaut +500,000 EUR pour toi (la banque te doit de l'argent).

\textbf{Scénario 1 : Tout va bien}\\
La banque XYZ paie ce qu'elle doit. Tu reçois tes 500,000 EUR.

\textbf{Scénario 2 : La banque fait faillite}\\
La banque XYZ fait défaut. Tu ne reçois que 40\% de ce qu'on te doit (taux de recouvrement).\\
Tu perds : $500\,000 \times (1 - 40\%) = 300\,000$ EUR
\end{exemple}

\begin{attention}{Point Clé}
Le risque de contrepartie n'existe que si le contrat a une \textbf{valeur positive pour nous}.

Si le contrat vaut -500,000 EUR (nous devons de l'argent), c'est \textbf{l'autre} qui a un risque sur nous !
\end{attention}

\subsection{Deux Termes Importants}

\begin{definition}{LGD - Loss Given Default}
\textbf{LGD} (Loss Given Default) = Perte en cas de défaut.

C'est le pourcentage qu'on perd si la contrepartie fait faillite.

$$\text{LGD} = 1 - \text{Taux de Recouvrement}$$

\textbf{Exemple :} Si on récupère 40\% en cas de faillite, LGD = 60\%.
\end{definition}

\begin{definition}{Hazard Rate (Taux de Défaut)}
Le \textbf{hazard rate} ($\lambda$) est la probabilité \textbf{instantanée} de défaut.

\textbf{Exemple simplifié :} $\lambda = 1.2\%$ par an signifie environ 1.2\% de chance de défaut chaque année.

La \textbf{probabilité de survie} jusqu'au temps $t$ est :
$$S(t) = e^{-\lambda \times t}$$
\end{definition}

\begin{exemple}{Calcul de Probabilité de Survie}
Contrepartie avec $\lambda = 2\%$ par an.

\textbf{Probabilité de survie à 5 ans :}
$$S(5) = e^{-0.02 \times 5} = e^{-0.10} \approx 90.5\%$$

\textbf{Probabilité de défaut dans les 5 ans :}
$$PD(5) = 1 - S(5) = 1 - 90.5\% = 9.5\%$$
\end{exemple}

\newpage
% ============================================================================
% CHAPITRE 3 : EXPOSITION
% ============================================================================
\section{L'Exposition : Combien On Peut Perdre ?}

\subsection{Le Problème}

La valeur d'un dérivé \textbf{change tous les jours} car les taux bougent. Donc notre risque change aussi !

\begin{center}
\begin{tikzpicture}[scale=0.8]
    \draw[->] (0,0) -- (12,0) node[right] {Temps};
    \draw[->] (0,-2) -- (0,3) node[above] {Valeur du swap};
    \draw[dashed] (0,0) -- (12,0);

    \draw[very thick, darkblue] plot[smooth] coordinates {
        (0,0.5) (1,1) (2,0.8) (3,1.5) (4,2) (5,1.8) (6,1) (7,0.5) (8,-0.5) (9,0.3) (10,0.8) (11,0.2)
    };

    \fill[lightgreen, opacity=0.3] (0,0) -- plot[smooth] coordinates {
        (0,0.5) (1,1) (2,0.8) (3,1.5) (4,2) (5,1.8) (6,1) (7,0.5) (8,0)
    } -- (8,0) -- cycle;

    \fill[lightred, opacity=0.3] (8,0) -- plot[smooth] coordinates {
        (8,-0.5) (9,0)
    } -- (9,0) -- cycle;

    \node[darkgreen] at (4,2.5) {Valeur positive = ON a un risque};
    \node[darkred] at (8,-1.5) {Valeur négative = EUX ont un risque};
\end{tikzpicture}
\end{center}

\subsection{Les Métriques d'Exposition}

Comme on ne sait pas ce que le futur nous réserve, on \textbf{simule des milliers de scénarios} et on calcule des statistiques.

\begin{definition}{EPE - Expected Positive Exposure}
\textbf{EPE} = Exposition Positive Moyenne

C'est la \textbf{moyenne} de notre exposition quand elle est positive (= quand on a un risque).

$$\text{EPE}(t) = \mathbb{E}[\max(V_t, 0)]$$

\textbf{En français :} "En moyenne, combien nous doit la contrepartie à la date $t$ ?"
\end{definition}

\begin{definition}{ENE - Expected Negative Exposure}
\textbf{ENE} = Exposition Négative Moyenne

C'est la \textbf{moyenne} de notre exposition quand elle est négative (= quand c'est nous qui devons).

$$\text{ENE}(t) = \mathbb{E}[\max(-V_t, 0)]$$

\textbf{En français :} "En moyenne, combien devons-nous à la contrepartie à la date $t$ ?"
\end{definition}

\begin{definition}{PFE - Potential Future Exposure}
\textbf{PFE} = Exposition Future Potentielle

C'est l'exposition dans le \textbf{pire des cas} (souvent 95\% ou 99\%).

$$\text{PFE}_{95\%}(t) = \text{Quantile}_{95\%}[\max(V_t, 0)]$$

\textbf{En français :} "Dans 95\% des scénarios, notre exposition sera inférieure à cette valeur."
\end{definition}

\begin{exemple}{Comprendre EPE, ENE, PFE}
On simule 1000 scénarios pour un swap. À la date t = 2 ans, voici les valeurs du swap :

\begin{center}
\begin{tabular}{lc}
\toprule
Scénario & Valeur du swap \\
\midrule
1 & +500,000 \\
2 & -200,000 \\
3 & +800,000 \\
... & ... \\
1000 & +300,000 \\
\bottomrule
\end{tabular}
\end{center}

\textbf{Supposons que parmi les 1000 scénarios :}
\begin{itemize}
    \item 600 ont une valeur positive (moyenne des positifs = +400,000)
    \item 400 ont une valeur négative (moyenne des négatifs = -250,000)
\end{itemize}

\textbf{Calculs :}
\begin{itemize}
    \item $\text{EPE} = \frac{600 \times 400\,000 + 400 \times 0}{1000} = 240\,000$ EUR
    \item $\text{ENE} = \frac{600 \times 0 + 400 \times 250\,000}{1000} = 100\,000$ EUR
    \item $\text{PFE}_{95\%}$ = la 950ème plus grande valeur positive
\end{itemize}
\end{exemple}

\newpage
% ============================================================================
% CHAPITRE 4 : MONTE CARLO
% ============================================================================
\section{La Simulation Monte Carlo}

\subsection{Pourquoi Simuler ?}

\begin{attention}{Le Problème}
On ne connaît pas le futur ! Les taux d'intérêt et le taux de change peuvent aller dans n'importe quelle direction.

\textbf{Solution :} On simule des \textbf{milliers de futurs possibles} et on fait des statistiques.
\end{attention}

\subsection{Comment ça Marche ?}

\begin{definition}{Simulation Monte Carlo}
\textbf{Monte Carlo} = Générer des milliers de scénarios aléatoires pour estimer des valeurs qu'on ne peut pas calculer exactement.

\textbf{Dans notre cas :}
\begin{enumerate}
    \item On génère 5000 trajectoires possibles des taux d'intérêt
    \item On génère 5000 trajectoires possibles du taux de change
    \item Pour chaque trajectoire, on calcule la valeur du portefeuille à chaque date
    \item On fait la moyenne pour obtenir EPE, ENE, etc.
\end{enumerate}
\end{definition}

\begin{center}
\begin{tikzpicture}[scale=0.7]
    \draw[->] (0,0) -- (10,0) node[right] {Temps};
    \draw[->] (0,-1) -- (0,4) node[above] {Taux};

    % Plusieurs trajectoires
    \draw[blue, opacity=0.3] plot[smooth] coordinates {(0,2) (2,2.3) (4,2.1) (6,2.5) (8,2.8)};
    \draw[blue, opacity=0.3] plot[smooth] coordinates {(0,2) (2,1.8) (4,1.5) (6,1.7) (8,1.3)};
    \draw[blue, opacity=0.3] plot[smooth] coordinates {(0,2) (2,2.1) (4,2.4) (6,2.2) (8,2.6)};
    \draw[blue, opacity=0.3] plot[smooth] coordinates {(0,2) (2,1.9) (4,2.0) (6,1.8) (8,1.5)};
    \draw[blue, opacity=0.3] plot[smooth] coordinates {(0,2) (2,2.2) (4,2.6) (6,3.0) (8,3.2)};
    \draw[blue, opacity=0.3] plot[smooth] coordinates {(0,2) (2,1.7) (4,1.4) (6,1.2) (8,0.8)};
    \draw[blue, opacity=0.3] plot[smooth] coordinates {(0,2) (2,2.0) (4,1.9) (6,2.1) (8,2.0)};

    \draw[red, very thick] plot[smooth] coordinates {(0,2) (2,2.0) (4,2.0) (6,2.0) (8,2.0)};

    \node at (5,-1.5) {5000 trajectoires simulées};
    \node[red] at (9.5,2) {Moyenne};

    \fill (0,2) circle (3pt);
    \node[below] at (0,2) {Aujourd'hui};
\end{tikzpicture}
\end{center}

\subsection{Les Modèles Utilisés}

\subsubsection{Pour les Taux d'Intérêt : Ornstein-Uhlenbeck}

\begin{formule}{Modèle Ornstein-Uhlenbeck (Vasicek)}
$$dr = \kappa(\theta - r) \, dt + \sigma \, dW$$

\textbf{Paramètres :}
\begin{itemize}
    \item $r$ = taux d'intérêt actuel
    \item $\kappa$ = vitesse de retour à la moyenne (ex: 0.1)
    \item $\theta$ = taux moyen long terme (ex: 2\%)
    \item $\sigma$ = volatilité (ex: 1\%)
    \item $dW$ = choc aléatoire (mouvement brownien)
\end{itemize}
\end{formule}

\begin{exemple}{Intuition du Modèle OU}
\textbf{Si le taux actuel est 3\% et $\theta = 2\%$ :}

Le terme $\kappa(\theta - r) = \kappa(2\% - 3\%) < 0$ va \textbf{tirer le taux vers le bas}.

$\Rightarrow$ Le modèle OU fait que les taux \textbf{reviennent vers leur moyenne} à long terme.

C'est réaliste : les taux ne montent pas à l'infini ni ne descendent à -100\%.
\end{exemple}

\subsubsection{Pour le Taux de Change : GBM}

\begin{formule}{Modèle GBM (Geometric Brownian Motion)}
$$\frac{dS}{S} = (r_d - r_f) \, dt + \sigma \, dW$$

\textbf{Paramètres :}
\begin{itemize}
    \item $S$ = taux de change (ex: EUR/USD = 1.10)
    \item $r_d$ = taux domestique (USD)
    \item $r_f$ = taux étranger (EUR)
    \item $\sigma$ = volatilité FX (ex: 12\%)
\end{itemize}
\end{formule}

\newpage
% ============================================================================
% CHAPITRE 5 : COLLATÉRAL
% ============================================================================
\section{Le Collatéral : Réduire le Risque}

\subsection{C'est quoi le Collatéral ?}

\begin{definition}{Collatéral (Garantie)}
Le \textbf{collatéral} est une garantie (souvent du cash) qu'on échange pour \textbf{réduire le risque de contrepartie}.

\textbf{Idée :} Si tu me dois 1 million et que tu me donnes 1 million en garantie, même si tu fais faillite, j'ai déjà l'argent !
\end{definition}

\subsection{Variation Margin (VM)}

\begin{definition}{Variation Margin}
La \textbf{Variation Margin} est le collatéral échangé \textbf{régulièrement} pour couvrir la valeur de marché actuelle du contrat.

\textbf{Règle simple :}
\begin{itemize}
    \item Si le contrat vaut +1M pour moi $\Rightarrow$ la contrepartie me donne 1M
    \item Si le contrat vaut -1M pour moi $\Rightarrow$ je donne 1M à la contrepartie
\end{itemize}
\end{definition}

\begin{exemple}{Variation Margin en Action}
\textbf{Jour 1 :} Swap vaut 0 $\Rightarrow$ pas d'échange

\textbf{Jour 30 :} Swap vaut +500k pour nous $\Rightarrow$ on reçoit 500k de collatéral

\textbf{Jour 60 :} Swap vaut +300k pour nous $\Rightarrow$ on rend 200k (on garde 300k)

\textbf{Jour 90 :} Swap vaut -100k pour nous $\Rightarrow$ on rend tout et on paie 100k
\end{exemple}

\subsection{Paramètres du Collatéral}

\begin{definition}{Threshold (Seuil)}
Le \textbf{seuil} est le montant d'exposition en dessous duquel on n'échange pas de collatéral.

\textbf{Exemple :} Seuil = 1M EUR

Si exposition = 800k $\Rightarrow$ pas de collatéral (en dessous du seuil)\\
Si exposition = 1.5M $\Rightarrow$ on reçoit 500k de collatéral (1.5M - 1M)
\end{definition}

\begin{definition}{MTA - Minimum Transfer Amount}
Le \textbf{MTA} est le montant minimum pour faire un transfert de collatéral.

\textbf{Exemple :} MTA = 100k EUR

Si la variation depuis le dernier appel est de 50k $\Rightarrow$ pas de transfert\\
Si la variation est de 150k $\Rightarrow$ transfert de 150k
\end{definition}

\begin{definition}{MPR - Margin Period of Risk}
Le \textbf{MPR} est le temps qu'il faut pour clôturer une position si la contrepartie fait défaut.

\textbf{Typiquement :} 10 jours ouvrés

\textbf{Pourquoi c'est important ?}\\
Pendant ces 10 jours, le marché peut bouger et on peut perdre de l'argent même si on avait du collatéral avant.
\end{definition}

\begin{exemple}{Impact du MPR}
\textbf{Jour 0 :} Swap vaut +1M, on a 1M de collatéral $\Rightarrow$ exposition nette = 0

\textbf{La contrepartie fait défaut !}

\textbf{Pendant les 10 jours de clôture :}
\begin{itemize}
    \item Le swap peut passer de +1M à +1.5M
    \item On a toujours seulement 1M de collatéral
    \item Exposition réelle = 1.5M - 1M = \textbf{500k de risque}
\end{itemize}

$\Rightarrow$ Le MPR crée un "risque résiduel" même avec du collatéral.
\end{exemple}

\subsection{Initial Margin (IM)}

\begin{definition}{Initial Margin}
L'\textbf{Initial Margin} est du collatéral supplémentaire pour couvrir le risque pendant le MPR.

\textbf{Calcul simplifié :}
$$\text{IM} \approx 1.5 \times \text{EPE}$$

C'est de l'argent "bloqué" qui coûte (d'où le MVA qu'on verra plus tard).
\end{definition}

\newpage
% ============================================================================
% CHAPITRE 6 : LES xVA
% ============================================================================
\section{Les xVA : Les Ajustements de Valorisation}

\subsection{Vue d'Ensemble}

\begin{resume}{Les 5 xVA}
\begin{center}
\begin{tabular}{llll}
\toprule
\textbf{xVA} & \textbf{Nom} & \textbf{C'est quoi ?} & \textbf{Coût/Bénéfice} \\
\midrule
CVA & Credit VA & Risque de défaut de la contrepartie & Coût (+) \\
DVA & Debt VA & Risque de notre propre défaut & Bénéfice (-) \\
FVA & Funding VA & Coût de financement & Coût (+) \\
MVA & Margin VA & Coût de la marge initiale & Coût (+) \\
KVA & Capital VA & Coût du capital réglementaire & Coût (+) \\
\bottomrule
\end{tabular}
\end{center}

\vspace{0.3cm}
$$\boxed{\text{Total xVA} = \text{CVA} - \text{DVA} + \text{FVA} + \text{MVA} + \text{KVA}}$$
\end{resume}

\subsection{CVA - Credit Valuation Adjustment}

\begin{definition}{CVA}
Le \textbf{CVA} est le coût du risque que la \textbf{contrepartie fasse défaut}.

\textbf{Intuition :} C'est comme une "prime d'assurance" contre le défaut de la contrepartie.
\end{definition}

\begin{formule}{CVA}
$$\text{CVA} = \sum_{i=1}^{n} DF(t_i) \times EPE(t_i) \times LGD \times \Delta PD(t_i)$$

\textbf{Où :}
\begin{itemize}
    \item $DF(t_i)$ = facteur d'actualisation (valeur aujourd'hui de 1€ futur)
    \item $EPE(t_i)$ = exposition positive espérée à $t_i$
    \item $LGD$ = perte en cas de défaut (ex: 60\%)
    \item $\Delta PD(t_i)$ = probabilité de défaut entre $t_{i-1}$ et $t_i$
\end{itemize}
\end{formule}

\begin{exemple}{Calcul de CVA Simplifié}
\textbf{Données :}
\begin{itemize}
    \item Horizon : 2 ans (2 périodes)
    \item EPE année 1 = 1M EUR, EPE année 2 = 800k EUR
    \item DF = 0.98 (année 1), 0.96 (année 2)
    \item LGD = 60\%
    \item $\lambda$ = 2\% par an $\Rightarrow$ $\Delta PD_1 \approx 2\%$, $\Delta PD_2 \approx 1.96\%$
\end{itemize}

\textbf{Calcul :}
\begin{align*}
\text{CVA} &= 0.98 \times 1\,000\,000 \times 0.6 \times 0.02 \\
           &+ 0.96 \times 800\,000 \times 0.6 \times 0.0196 \\
           &= 11\,760 + 9\,031 \\
           &= \textbf{20\,791 EUR}
\end{align*}

\textbf{Interprétation :} On devrait charger ~21k EUR au client pour couvrir le risque de défaut.
\end{exemple}

\subsection{DVA - Debt Valuation Adjustment}

\begin{definition}{DVA}
Le \textbf{DVA} est le "bénéfice" lié au fait que \textbf{nous pouvons aussi faire défaut}.

\textbf{Logique (controversée) :} Si nous faisons défaut, nous ne paierons pas ce que nous devons. C'est un "gain" pour nous.

Le DVA est calculé sur l'\textbf{ENE} (quand nous devons de l'argent).
\end{definition}

\begin{formule}{DVA}
$$\text{DVA} = \sum_{i=1}^{n} DF(t_i) \times ENE(t_i) \times LGD_{\text{own}} \times \Delta PD_{\text{own}}(t_i)$$
\end{formule}

\begin{attention}{Controverse du DVA}
Le DVA est controversé car il dit : "Plus je suis risqué, plus j'ai de bénéfices !"

C'est contre-intuitif, mais c'est la logique comptable IFRS 13 / FAS 157.

Dans le Total xVA, on le \textbf{soustrait} : Total = CVA \textbf{- DVA} + ...
\end{attention}

\subsection{FVA - Funding Valuation Adjustment}

\begin{definition}{FVA}
Le \textbf{FVA} est le coût de \textbf{financement} de l'exposition positive.

\textbf{Intuition :} Si le swap vaut +1M pour nous, la contrepartie nous "doit" 1M. Mais tant qu'on ne l'a pas reçu, on doit peut-être emprunter cet argent sur le marché.
\end{definition}

\begin{formule}{FVA}
$$\text{FVA} = \sum_{i=1}^{n} DF(t_i) \times EPE(t_i) \times s_f \times \Delta t$$

\textbf{Où :}
\begin{itemize}
    \item $s_f$ = spread de funding (ex: 1\% = 100 bps au-dessus du taux sans risque)
    \item $\Delta t$ = durée de la période
\end{itemize}
\end{formule}

\begin{exemple}{Calcul de FVA}
\textbf{Données :}
\begin{itemize}
    \item EPE moyen sur 5 ans = 2M EUR
    \item Spread de funding = 1\% par an
\end{itemize}

\textbf{Approximation simple :}
$$\text{FVA} \approx 2\,000\,000 \times 1\% \times 5 \times 0.9 \approx \textbf{90\,000 EUR}$$

(Le 0.9 est une approximation du facteur d'actualisation moyen)
\end{exemple}

\subsection{MVA - Margin Valuation Adjustment}

\begin{definition}{MVA}
Le \textbf{MVA} est le coût de \textbf{financement de l'Initial Margin}.

\textbf{Intuition :} L'IM est de l'argent bloqué qu'on ne peut pas utiliser. On doit l'emprunter, donc ça coûte.
\end{definition}

\begin{formule}{MVA}
$$\text{MVA} = \sum_{i=1}^{n} DF(t_i) \times IM(t_i) \times s_f \times \Delta t$$

\textbf{Similaire au FVA, mais sur l'IM au lieu de l'EPE.}
\end{formule}

\subsection{KVA - Capital Valuation Adjustment}

\begin{definition}{KVA}
Le \textbf{KVA} est le coût du \textbf{capital réglementaire} que la banque doit détenir.

\textbf{Intuition :} Les régulateurs obligent les banques à garder du capital pour absorber les pertes potentielles. Ce capital a un coût d'opportunité.
\end{definition}

\begin{formule}{KVA}
$$\text{KVA} = \sum_{i=1}^{n} DF(t_i) \times K(t_i) \times CoC \times \Delta t$$

\textbf{Où :}
\begin{itemize}
    \item $K(t_i)$ = capital requis = $\beta \times EAD(t_i)$
    \item $\beta$ = ratio de capital (ex: 8\%)
    \item $CoC$ = coût du capital (ex: 10\%)
    \item $EAD$ = Exposure at Default (calculé via SA-CCR)
\end{itemize}
\end{formule}

\newpage
% ============================================================================
% CHAPITRE 7 : SA-CCR
% ============================================================================
\section{SA-CCR : Le Capital Réglementaire}

\subsection{C'est quoi SA-CCR ?}

\begin{definition}{SA-CCR}
\textbf{SA-CCR} = Standardized Approach for Counterparty Credit Risk

C'est la méthode \textbf{réglementaire} (Bâle III) pour calculer l'\textbf{Exposure at Default (EAD)} des produits dérivés.

L'EAD sert ensuite à calculer le capital réglementaire que la banque doit détenir.
\end{definition}

\subsection{La Formule SA-CCR}

\begin{formule}{SA-CCR}
$$\boxed{\text{EAD} = \alpha \times (\text{RC} + \text{PFE})}$$

\textbf{Où :}
\begin{itemize}
    \item $\alpha = 1.4$ (multiplicateur réglementaire, fixé par Bâle)
    \item $\text{RC}$ = Replacement Cost (coût de remplacement)
    \item $\text{PFE}$ = Potential Future Exposure
\end{itemize}
\end{formule}

\subsubsection{Replacement Cost (RC)}

\begin{definition}{Replacement Cost}
$$\text{RC} = \max(V - C, 0)$$

\textbf{Où :}
\begin{itemize}
    \item $V$ = valeur de marché actuelle du portefeuille
    \item $C$ = collatéral détenu
\end{itemize}

\textbf{Intuition :} C'est ce qu'on perdrait si la contrepartie faisait défaut \textbf{aujourd'hui}.
\end{definition}

\subsubsection{Potential Future Exposure (PFE)}

\begin{definition}{PFE dans SA-CCR}
$$\text{PFE} = \text{multiplier} \times \text{AddOn}$$

\textbf{AddOn} dépend du type de produit :
\begin{itemize}
    \item \textbf{Taux d'intérêt :} AddOn = Notionnel $\times$ SF $\times$ MF
    \item \textbf{FX :} AddOn = Notionnel $\times$ SF
\end{itemize}

\textbf{SF} = Supervisory Factor (facteur réglementaire) :
\begin{itemize}
    \item Taux d'intérêt : 0.5\%
    \item FX : 4\%
\end{itemize}

\textbf{MF} = Maturity Factor = $\sqrt{\min(M, 5)}$ pour les taux
\end{definition}

\begin{exemple}{Calcul SA-CCR}
\textbf{Portefeuille :}
\begin{itemize}
    \item 1 IRS : Notionnel = 10M, Maturité = 5 ans
    \item 1 FX Forward : Notionnel = 5M EUR (= 5.5M USD)
    \item Valeur de marché actuelle : V = +800k USD
    \item Pas de collatéral
\end{itemize}

\textbf{Étape 1 : RC}
$$\text{RC} = \max(800\,000 - 0, 0) = 800\,000$$

\textbf{Étape 2 : AddOn IRS}
$$\text{AddOn}_{\text{IRS}} = 10\,000\,000 \times 0.5\% \times \sqrt{5} = 111\,803$$

\textbf{Étape 3 : AddOn FX}
$$\text{AddOn}_{\text{FX}} = 5\,500\,000 \times 4\% = 220\,000$$

\textbf{Étape 4 : PFE} (multiplier = 1 car pas de collatéral excédentaire)
$$\text{PFE} = 1 \times (111\,803 + 220\,000) = 331\,803$$

\textbf{Étape 5 : EAD}
$$\text{EAD} = 1.4 \times (800\,000 + 331\,803) = \textbf{1\,584\,524 USD}$$
\end{exemple}

\newpage
% ============================================================================
% CHAPITRE 8 : RÉCAPITULATIF
% ============================================================================
\section{Récapitulatif : Tout Relier}

\subsection{Le Flux de Calcul de l'Application}

\begin{center}
\begin{tikzpicture}[
    block/.style={draw, rounded corners, fill=lightblue, minimum width=4cm, minimum height=1cm, align=center},
    arrow/.style={->, very thick}
]

\node[block] (port) at (0,0) {1. Portefeuille\\(IRS, FX Forwards)};
\node[block] (mc) at (0,-2.5) {2. Monte Carlo\\(5000 simulations)};
\node[block] (exp) at (0,-5) {3. Expositions\\(EPE, ENE, PFE)};
\node[block] (coll) at (5,-5) {4. Collatéral\\(VM + IM)};
\node[block] (xva) at (5,-2.5) {5. xVA\\(CVA, DVA, FVA, MVA, KVA)};
\node[block] (saccr) at (5,0) {6. SA-CCR\\(EAD)};

\draw[arrow] (port) -- (mc);
\draw[arrow] (mc) -- (exp);
\draw[arrow] (exp) -- (coll);
\draw[arrow] (coll) -- (xva);
\draw[arrow] (xva) -- (saccr);
\draw[arrow, dashed] (exp) -- (xva);

\end{tikzpicture}
\end{center}

\subsection{Tableau Récapitulatif des Termes}

\begin{center}
\small
\begin{tabular}{llp{7cm}}
\toprule
\textbf{Terme} & \textbf{Signification} & \textbf{Explication simple} \\
\midrule
IRS & Interest Rate Swap & Échange de taux fixe contre taux variable \\
FX Forward & Foreign Exchange Forward & Achat/vente de devise à terme \\
EPE & Expected Positive Exposure & Moyenne de ce qu'on peut perdre \\
ENE & Expected Negative Exposure & Moyenne de ce qu'on doit \\
PFE & Potential Future Exposure & Pire cas (95\% ou 99\%) \\
LGD & Loss Given Default & Perte en cas de défaut (ex: 60\%) \\
Hazard Rate & Taux de défaut & Probabilité instantanée de défaut \\
VM & Variation Margin & Collatéral quotidien \\
IM & Initial Margin & Collatéral de sécurité supplémentaire \\
MPR & Margin Period of Risk & Temps pour clôturer (10 jours) \\
MTA & Minimum Transfer Amount & Montant minimum d'appel de marge \\
CVA & Credit VA & Coût du risque de défaut contrepartie \\
DVA & Debt VA & "Bénéfice" de notre propre risque \\
FVA & Funding VA & Coût de financement \\
MVA & Margin VA & Coût de l'Initial Margin \\
KVA & Capital VA & Coût du capital réglementaire \\
SA-CCR & Standardized Approach CCR & Méthode réglementaire pour EAD \\
EAD & Exposure at Default & Exposition pour le capital \\
\bottomrule
\end{tabular}
\end{center}

\subsection{Les Formules Clés}

\begin{resume}{Formules à Retenir}

\textbf{Exposition :}
$$\text{EPE}(t) = \mathbb{E}[\max(V_t, 0)] \qquad \text{ENE}(t) = \mathbb{E}[\max(-V_t, 0)]$$

\textbf{Probabilité de survie :}
$$S(t) = e^{-\lambda t}$$

\textbf{xVA :}
$$\text{CVA} = \sum_i DF_i \times EPE_i \times LGD \times \Delta PD_i$$
$$\text{FVA} = \sum_i DF_i \times EPE_i \times s_f \times \Delta t$$
$$\text{Total xVA} = \text{CVA} - \text{DVA} + \text{FVA} + \text{MVA} + \text{KVA}$$

\textbf{SA-CCR :}
$$\text{EAD} = 1.4 \times (\text{RC} + \text{PFE})$$
\end{resume}

\newpage
% ============================================================================
% ANNEXE : GLOSSAIRE RAPIDE POUR LA PRÉSENTATION
% ============================================================================
\section{Annexe : Aide-Mémoire pour ta Présentation}

\subsection{Si on te pose une question...}

\begin{center}
\begin{tabular}{p{5cm}p{9cm}}
\toprule
\textbf{Question} & \textbf{Réponse simple} \\
\midrule
"C'est quoi un xVA ?" & "Ce sont les coûts cachés des dérivés : risque de défaut, financement, marge, capital." \\
\midrule
"Pourquoi 5000 simulations ?" & "C'est un compromis entre précision et temps de calcul. Plus on en fait, plus c'est précis." \\
\midrule
"C'est quoi Monte Carlo ?" & "On simule des milliers de futurs possibles pour estimer des moyennes et des risques." \\
\midrule
"Pourquoi le DVA est négatif ?" & "C'est un bénéfice : si on fait défaut, on ne paie pas ce qu'on doit. C'est controversé mais c'est la norme comptable." \\
\midrule
"C'est quoi le collatéral ?" & "C'est une garantie en cash qu'on échange pour réduire le risque. Si tu me dois 1M et me donnes 1M de garantie, je n'ai plus de risque." \\
\midrule
"C'est quoi SA-CCR ?" & "C'est la méthode réglementaire Bâle III pour calculer combien de capital la banque doit garder." \\
\midrule
"C'est quoi $\alpha = 1.4$ ?" & "C'est un multiplicateur de sécurité fixé par les régulateurs pour être conservateur." \\
\midrule
"Pourquoi l'EPE diminue avec le collatéral ?" & "Parce que le collatéral couvre une partie de l'exposition. On ne risque que la partie non couverte." \\
\bottomrule
\end{tabular}
\end{center}

\subsection{Chiffres Typiques à Retenir}

\begin{itemize}
    \item \textbf{LGD} : 60\% (on récupère 40\% en cas de défaut)
    \item \textbf{Hazard rate} : 1-2\% par an pour une contrepartie investment grade
    \item \textbf{Funding spread} : 100 bps (1\%)
    \item \textbf{Coût du capital} : 10\%
    \item \textbf{MPR} : 10 jours ouvrés
    \item \textbf{Seuil collatéral} : souvent 0 à 10M selon l'accord
    \item \textbf{$\alpha$ SA-CCR} : 1.4 (fixé par Bâle)
    \item \textbf{SF taux} : 0.5\%
    \item \textbf{SF FX} : 4\%
\end{itemize}

\end{document}
