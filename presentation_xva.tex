\documentclass[aspectratio=169,12pt]{beamer}

% ============================================================================
% THEME ET COULEURS
% ============================================================================
\usetheme{Madrid}
\usecolortheme{whale}

% Couleurs personnalisées
\definecolor{darkblue}{RGB}{0, 51, 102}
\definecolor{lightblue}{RGB}{230, 242, 255}
\definecolor{accentgreen}{RGB}{0, 128, 0}
\definecolor{accentred}{RGB}{180, 0, 0}

\setbeamercolor{title}{fg=white,bg=darkblue}
\setbeamercolor{frametitle}{fg=white,bg=darkblue}
\setbeamercolor{block title}{fg=white,bg=darkblue}
\setbeamercolor{block body}{fg=black,bg=lightblue}

% Pas de navigation
\setbeamertemplate{navigation symbols}{}

% Numéros de slides
\setbeamertemplate{footline}[frame number]

% ============================================================================
% PACKAGES
% ============================================================================
\usepackage[utf8]{inputenc}
\usepackage[T1]{fontenc}
\usepackage[french]{babel}
\usepackage{tikz}
\usepackage{graphicx}
\usepackage{booktabs}
\usepackage{fontawesome5}

% ============================================================================
% INFOS
% ============================================================================
\title{\textbf{xVA Calculation Engine}}
\subtitle{Application Interactive de Calcul des Ajustements de Valorisation}
\author{Présentation de l'Application}
\institute{Projet xVA - Master Finance}
\date{}

% ============================================================================
% DOCUMENT
% ============================================================================
\begin{document}

% --------------------------------------------------------------------------
% SLIDE TITRE
% --------------------------------------------------------------------------
\begin{frame}
\titlepage
\end{frame}

% --------------------------------------------------------------------------
% SLIDE 1 : C'EST QUOI xVA ? (pour que TU comprennes)
% --------------------------------------------------------------------------
\begin{frame}{Rappel : C'est quoi les xVA ?}

\begin{block}{xVA = Coûts cachés des produits dérivés}
Quand une banque fait un trade, le prix "théorique" ne suffit pas. \\
Il faut ajouter des \textbf{ajustements} pour les vrais risques :
\end{block}

\vspace{0.5cm}

\begin{columns}[T]
\begin{column}{0.48\textwidth}
\textbf{\textcolor{accentred}{Coûts :}}
\begin{itemize}
    \item \textbf{CVA} -- Risque que la contrepartie fasse défaut
    \item \textbf{FVA} -- Coût de financement
    \item \textbf{MVA} -- Coût des marges (collatéral)
    \item \textbf{KVA} -- Coût du capital réglementaire
\end{itemize}
\end{column}

\begin{column}{0.48\textwidth}
\textbf{\textcolor{accentgreen}{Bénéfice :}}
\begin{itemize}
    \item \textbf{DVA} -- "Gain" si nous faisons défaut
\end{itemize}

\vspace{0.5cm}

\begin{block}{Formule}
\centering
\textbf{Total xVA = CVA - DVA + FVA + MVA + KVA}
\end{block}
\end{column}
\end{columns}

\end{frame}

% --------------------------------------------------------------------------
% SLIDE 2 : OBJECTIF DE L'APPLICATION
% --------------------------------------------------------------------------
\begin{frame}{Notre Application : Objectif}

\begin{block}{Problème}
Calculer les xVA à la main est \textbf{impossible} :
\begin{itemize}
    \item Il faut simuler des \textbf{milliers de scénarios} futurs
    \item Calculer l'exposition à \textbf{chaque date}
    \item Appliquer le \textbf{collatéral}
    \item Intégrer les \textbf{probabilités de défaut}
\end{itemize}
\end{block}

\vspace{0.3cm}

\begin{block}{Solution : Notre Application}
\begin{center}
\begin{tikzpicture}[scale=0.9]
    \node[draw, rounded corners, fill=lightblue, minimum width=2.5cm, minimum height=1cm] (input) at (0,0) {\textbf{Portefeuille}};
    \node[draw, rounded corners, fill=darkblue, text=white, minimum width=3cm, minimum height=1cm] (engine) at (5,0) {\textbf{Monte Carlo}};
    \node[draw, rounded corners, fill=lightblue, minimum width=2.5cm, minimum height=1cm] (output) at (10,0) {\textbf{xVA + SA-CCR}};

    \draw[->, very thick] (input) -- (engine);
    \draw[->, very thick] (engine) -- (output);

    \node[below] at (0,-0.8) {\small IRS, FX Fwd};
    \node[below] at (5,-0.8) {\small 5000 simulations};
    \node[below] at (10,-0.8) {\small Résultats};
\end{tikzpicture}
\end{center}
\end{block}

\end{frame}

% --------------------------------------------------------------------------
% SLIDE 3 : TECHNOLOGIES
% --------------------------------------------------------------------------
\begin{frame}{Technologies Utilisées}

\begin{columns}[T]
\begin{column}{0.55\textwidth}

\begin{block}{\faIcon{python} Stack Technique}
\begin{itemize}
    \item \textbf{Python 3.11+} -- Langage principal
    \item \textbf{NumPy / Pandas} -- Calculs matriciels
    \item \textbf{Streamlit} -- Interface web interactive
    \item \textbf{Plotly} -- Graphiques dynamiques
    \item \textbf{Pydantic} -- Validation des données
\end{itemize}
\end{block}

\begin{block}{\faIcon{calculator} Modèles Financiers}
\begin{itemize}
    \item \textbf{Ornstein-Uhlenbeck} -- Taux d'intérêt
    \item \textbf{GBM} -- Taux de change
    \item \textbf{Monte Carlo} -- Simulation
\end{itemize}
\end{block}

\end{column}

\begin{column}{0.42\textwidth}
\begin{block}{\faIcon{folder-open} Architecture}
\footnotesize
\texttt{xva-project/} \\
\texttt{├── xva\_core/} \\
\texttt{│\quad ├── market/} \textcolor{gray}{(modèles)} \\
\texttt{│\quad ├── instruments/} \textcolor{gray}{(IRS, FX)} \\
\texttt{│\quad ├── exposure/} \textcolor{gray}{(Monte Carlo)} \\
\texttt{│\quad ├── collateral/} \textcolor{gray}{(VM, IM)} \\
\texttt{│\quad ├── xva/} \textcolor{gray}{(CVA, DVA...)} \\
\texttt{│\quad └── reg/} \textcolor{gray}{(SA-CCR)} \\
\texttt{├── xva\_app/} \textcolor{gray}{(Streamlit)} \\
\texttt{└── tests/}
\end{block}
\end{column}
\end{columns}

\end{frame}

% --------------------------------------------------------------------------
% SLIDE 4 : INTERFACE - SIDEBAR
% --------------------------------------------------------------------------
\begin{frame}{Interface : Panneau de Configuration}

\begin{columns}[T]
\begin{column}{0.35\textwidth}
\begin{block}{\faIcon{sliders-h} Sidebar}
\small
Tous les paramètres sont modifiables :
\end{block}

\vspace{0.2cm}

\footnotesize
\textbf{\faIcon{dice} Monte Carlo}
\begin{itemize}
    \item Nombre de paths (100 - 10000)
    \item Horizon (1 - 10 ans)
    \item Pas de temps (mensuel/trimestriel)
\end{itemize}

\textbf{\faIcon{chart-line} Modèles de Marché}
\begin{itemize}
    \item Paramètres OU ($\kappa$, $\theta$, $\sigma$)
    \item Volatilité FX
    \item Corrélations
\end{itemize}

\end{column}

\begin{column}{0.35\textwidth}

\footnotesize
\textbf{\faIcon{shield-alt} Collatéral}
\begin{itemize}
    \item Seuil (ex: 1M\$)
    \item MTA (ex: 100K\$)
    \item MPR (ex: 10 jours)
\end{itemize}

\textbf{\faIcon{university} Crédit \& Funding}
\begin{itemize}
    \item LGD (ex: 60\%)
    \item Hazard rates ($\lambda$)
    \item Spread de funding
    \item Coût du capital
\end{itemize}

\vspace{0.3cm}

\begin{center}
\fbox{\textbf{\faIcon{play} Run}}
\end{center}

\end{column}

\begin{column}{0.28\textwidth}
\begin{block}{Valeurs par défaut}
\footnotesize
\begin{tabular}{ll}
\toprule
Paramètre & Valeur \\
\midrule
Paths & 5000 \\
Horizon & 5 ans \\
$\kappa$ & 0.10 \\
$\theta$ & 2\% \\
$\sigma$ & 100 bps \\
Vol FX & 12\% \\
Seuil & 1M\$ \\
MPR & 10 jours \\
LGD & 60\% \\
\bottomrule
\end{tabular}
\end{block}
\end{column}
\end{columns}

\end{frame}

% --------------------------------------------------------------------------
% SLIDE 5 : INTERFACE - ONGLETS
% --------------------------------------------------------------------------
\begin{frame}{Interface : Les 9 Onglets}

\begin{center}
\begin{tikzpicture}[
    tab/.style={draw, rounded corners, fill=lightblue, minimum width=2.8cm, minimum height=0.9cm, font=\footnotesize},
    desc/.style={font=\scriptsize, text width=2.6cm, align=center}
]

% Ligne 1
\node[tab] (t1) at (0,2) {\faIcon{briefcase} Portfolio};
\node[tab] (t2) at (3.2,2) {\faIcon{chart-area} Exposure};
\node[tab] (t3) at (6.4,2) {\faIcon{dollar-sign} xVA};
\node[tab] (t4) at (9.6,2) {\faIcon{landmark} SA-CCR};
\node[tab] (t5) at (12.8,2) {\faIcon{sliders-h} Calibration};

% Descriptions ligne 1
\node[desc] at (0,1) {Définir les trades IRS et FX};
\node[desc] at (3.2,1) {Profils EPE, ENE, PFE};
\node[desc] at (6.4,1) {CVA, DVA, FVA, MVA, KVA};
\node[desc] at (9.6,1) {Capital réglementaire};
\node[desc] at (12.8,1) {Calibrer les modèles};

% Ligne 2
\node[tab] (t6) at (1.6,-0.5) {\faIcon{bolt} Stress Test};
\node[tab] (t7) at (4.8,-0.5) {\faIcon{calculator} Sensitivités};
\node[tab] (t8) at (8,-0.5) {\faIcon{book} Méthodologie};
\node[tab] (t9) at (11.2,-0.5) {\faIcon{download} Export};

% Descriptions ligne 2
\node[desc] at (1.6,-1.5) {Scénarios extrêmes};
\node[desc] at (4.8,-1.5) {Greeks des xVA};
\node[desc] at (8,-1.5) {Formules et docs};
\node[desc] at (11.2,-1.5) {Excel, CSV, JSON};

\end{tikzpicture}
\end{center}

\vspace{0.3cm}

\begin{block}{Workflow Typique}
\centering
\textbf{Portfolio} $\rightarrow$ \textbf{Run} $\rightarrow$ \textbf{Exposure} $\rightarrow$ \textbf{xVA} $\rightarrow$ \textbf{SA-CCR} $\rightarrow$ \textbf{Export}
\end{block}

\end{frame}

% --------------------------------------------------------------------------
% SLIDE 6 : DÉMO - PORTEFEUILLE
% --------------------------------------------------------------------------
\begin{frame}{Démonstration : Portefeuille de Test}

\begin{columns}[T]
\begin{column}{0.48\textwidth}
\begin{block}{\faIcon{exchange-alt} Interest Rate Swaps (IRS)}
\begin{tabular}{lccc}
\toprule
\# & Notionnel & Taux & Maturité \\
\midrule
1 & 10M\$ & 2.5\% & 5 ans \\
2 & 15M\$ & 2.0\% & 3 ans \\
3 & 8M\$ & 3.0\% & 7 ans \\
\bottomrule
\end{tabular}

\vspace{0.3cm}
\small
\textit{Swap 1 \& 3 : payer fixe (pay fixed)} \\
\textit{Swap 2 : receveur fixe (receive fixed)}
\end{block}
\end{column}

\begin{column}{0.48\textwidth}
\begin{block}{\faIcon{euro-sign} FX Forwards}
\begin{tabular}{lccc}
\toprule
\# & Notionnel & Strike & Maturité \\
\midrule
1 & 5M EUR & 1.12 & 1 an \\
2 & 3M EUR & 1.08 & 2 ans \\
\bottomrule
\end{tabular}

\vspace{0.3cm}
\small
\textit{Forward 1 : achat EUR (buy foreign)} \\
\textit{Forward 2 : vente EUR (sell foreign)}
\end{block}
\end{column}
\end{columns}

\vspace{0.5cm}

\begin{center}
\begin{tikzpicture}
\node[draw, rounded corners, fill=darkblue, text=white, minimum width=8cm, minimum height=1cm] {
    \textbf{Notionnel Total $\approx$ 44 M\$}
};
\end{tikzpicture}
\end{center}

\end{frame}

% --------------------------------------------------------------------------
% SLIDE 7 : DÉMO - RÉSULTATS ATTENDUS
% --------------------------------------------------------------------------
\begin{frame}{Démonstration : Résultats Attendus}

\begin{columns}[T]
\begin{column}{0.48\textwidth}
\begin{block}{\faIcon{chart-line} Exposition}
\begin{itemize}
    \item \textbf{Peak EPE} $\approx$ 2.5M\$ (sans collatéral)
    \item \textbf{Peak EPE} $\approx$ 1.2M\$ (avec collatéral)
    \item Réduction $\approx$ \textbf{50\%} grâce au collatéral
\end{itemize}

\vspace{0.3cm}

\textbf{Métriques clés :}
\begin{itemize}
    \item EPE = Expected Positive Exposure
    \item ENE = Expected Negative Exposure
    \item PFE = Potential Future Exposure
\end{itemize}
\end{block}
\end{column}

\begin{column}{0.48\textwidth}
\begin{block}{\faIcon{dollar-sign} Breakdown xVA}
\begin{tabular}{lr}
\toprule
Composante & Valeur \\
\midrule
\textcolor{accentred}{CVA} & + coût \\
\textcolor{accentgreen}{DVA} & - bénéfice \\
\textcolor{accentred}{FVA} & + coût \\
\textcolor{accentred}{MVA} & + coût \\
\textcolor{accentred}{KVA} & + coût \\
\midrule
\textbf{Total xVA} & \textbf{XX bps} \\
\bottomrule
\end{tabular}
\end{block}

\begin{block}{\faIcon{landmark} SA-CCR}
$\text{EAD} = 1.4 \times (\text{RC} + \text{PFE})$
\end{block}
\end{column}
\end{columns}

\end{frame}

% --------------------------------------------------------------------------
% SLIDE 8 : CONCLUSION
% --------------------------------------------------------------------------
\begin{frame}{Récapitulatif}

\begin{block}{\faIcon{check-circle} Ce que permet l'application}
\begin{enumerate}
    \item \textbf{Définir} un portefeuille de dérivés (IRS, FX Forwards)
    \item \textbf{Simuler} l'évolution des marchés (Monte Carlo, 5000 paths)
    \item \textbf{Calculer} les expositions (EPE, ENE, PFE)
    \item \textbf{Appliquer} le collatéral (VM avec MPR, IM)
    \item \textbf{Obtenir} tous les xVA (CVA, DVA, FVA, MVA, KVA)
    \item \textbf{Calculer} le capital réglementaire SA-CCR
    \item \textbf{Exporter} les résultats (Excel, CSV, JSON)
\end{enumerate}
\end{block}

\vspace{0.3cm}

\begin{center}
\begin{tikzpicture}
\node[draw, rounded corners, fill=accentgreen, text=white, minimum width=10cm, minimum height=1.2cm, font=\large] {
    \textbf{Passons à la démonstration live !}
};
\end{tikzpicture}
\end{center}

\end{frame}

\end{document}
