\documentclass[11pt,a4paper]{article}

\usepackage[utf8]{inputenc}
\usepackage[T1]{fontenc}
\usepackage[french]{babel}
\usepackage[margin=2cm]{geometry}
\usepackage{xcolor}
\usepackage{tcolorbox}
\usepackage{enumitem}
\usepackage{fancyhdr}

% Couleurs
\definecolor{darkblue}{RGB}{0, 51, 102}
\definecolor{lightblue}{RGB}{230, 242, 255}
\definecolor{darkgreen}{RGB}{0, 100, 0}
\definecolor{lightgreen}{RGB}{230, 255, 230}
\definecolor{darkorange}{RGB}{200, 100, 0}
\definecolor{lightorange}{RGB}{255, 245, 230}

% Boîtes
\newtcolorbox{slide}[1]{
    colback=lightblue,
    colframe=darkblue,
    title={\textbf{#1}},
    fonttitle=\large
}

\newtcolorbox{texte}{
    colback=white,
    colframe=darkblue,
    left=5pt,
    right=5pt
}

\newtcolorbox{action}[1]{
    colback=lightorange,
    colframe=darkorange,
    title={\textbf{ACTION : #1}}
}

\newtcolorbox{conseil}{
    colback=lightgreen,
    colframe=darkgreen,
    title={\textbf{CONSEIL}}
}

% En-tête
\pagestyle{fancy}
\fancyhf{}
\fancyhead[L]{\textcolor{darkblue}{\textbf{Script Partie 1 - Chef d'Orchestre / Démo}}}
\fancyhead[R]{\textcolor{darkblue}{\thepage}}
\renewcommand{\headrulewidth}{0.5pt}

\begin{document}

% ============================================================================
% TITRE
% ============================================================================
\begin{center}
{\Huge\textbf{\textcolor{darkblue}{Script de Présentation}}}\\[0.5cm]
{\Large Partie 1 : Application \& Démonstration}\\[1cm]
{\large Durée totale : \textbf{6-7 minutes}}\\[0.5cm]
\end{center}

\begin{conseil}
\textbf{Avant de commencer :}
\begin{itemize}
    \item Ouvre l'application Streamlit AVANT la présentation
    \item Terminal : \texttt{cd "/Users/.../xva-project" \&\& streamlit run xva\_app/app.py}
    \item Laisse l'app ouverte en arrière-plan
    \item Garde ce script à côté de toi (sur ton téléphone ou imprimé)
\end{itemize}
\end{conseil}

\newpage

% ============================================================================
% SLIDE 1 : TITRE
% ============================================================================
\begin{slide}{SLIDE 1 -- Page de Titre}
\textit{Durée : 10 secondes}
\end{slide}

\begin{texte}
\textbf{Ce que tu dis :}

\og Bonjour à tous. Nous allons vous présenter notre projet : le \textbf{xVA Calculation Engine}, une application de calcul des ajustements de valorisation et du capital réglementaire pour les produits dérivés. \fg{}
\end{texte}

\begin{action}{Passe à la slide suivante}
\end{action}

\vspace{1cm}

% ============================================================================
% SLIDE 2 : PLAN
% ============================================================================
\begin{slide}{SLIDE 2 -- Plan de la Présentation}
\textit{Durée : 30 secondes}
\end{slide}

\begin{texte}
\textbf{Ce que tu dis :}

\og Voici le plan de notre présentation, divisée en 5 parties :

\begin{itemize}
    \item \textbf{Partie 1}, que je vais présenter : l'application et une démonstration live
    \item \textbf{Partie 2} : les modèles de marché utilisés pour simuler les taux et le change
    \item \textbf{Partie 3} : les instruments financiers et leur valorisation
    \item \textbf{Partie 4} : le calcul de l'exposition et la gestion du collatéral
    \item \textbf{Partie 5} : le calcul des xVA et le capital réglementaire SA-CCR
\end{itemize}

La présentation durera environ 25 à 30 minutes. \fg{}
\end{texte}

\begin{action}{Passe à la slide suivante}
\end{action}

\newpage

% ============================================================================
% SLIDE 3 : TITRE PARTIE 1
% ============================================================================
\begin{slide}{SLIDE 3 -- Titre Partie 1 : Application \& Démo}
\textit{Durée : 5 secondes}
\end{slide}

\begin{texte}
\textbf{Ce que tu dis :}

\og Commençons par la \textbf{Partie 1} : présentation de l'application et démonstration. \fg{}
\end{texte}

\begin{action}{Passe à la slide suivante}
\end{action}

\vspace{1cm}

% ============================================================================
% SLIDE 4 : RAPPEL xVA
% ============================================================================
\begin{slide}{SLIDE 4 -- Rappel : C'est quoi les xVA ?}
\textit{Durée : 1 minute}
\end{slide}

\begin{texte}
\textbf{Ce que tu dis :}

\og Avant de présenter l'application, un petit rappel sur ce que sont les \textbf{xVA}.

Quand une banque fait un trade sur un produit dérivé, le prix théorique ne suffit pas. Il faut ajouter des \textbf{ajustements} pour prendre en compte les vrais risques.

Ces ajustements sont principalement des \textbf{coûts} :

\begin{itemize}
    \item Le \textbf{CVA}, Credit Valuation Adjustment : c'est le coût du risque que la contrepartie fasse défaut
    \item Le \textbf{FVA}, Funding Valuation Adjustment : c'est le coût de financement de l'exposition
    \item Le \textbf{MVA}, Margin Valuation Adjustment : c'est le coût des marges, du collatéral
    \item Le \textbf{KVA}, Capital Valuation Adjustment : c'est le coût du capital réglementaire que la banque doit détenir
\end{itemize}

Il y a aussi un \textbf{bénéfice} :
\begin{itemize}
    \item Le \textbf{DVA}, Debt Valuation Adjustment : c'est le gain théorique si c'est nous qui faisons défaut
\end{itemize}

La formule clé est donc : \textbf{Total xVA = CVA moins DVA plus FVA plus MVA plus KVA}.

Mes collègues expliqueront chacun de ces termes en détail dans la partie 5. \fg{}
\end{texte}

\begin{action}{Passe à la slide suivante}
\end{action}

\newpage

% ============================================================================
% SLIDE 5 : OBJECTIF
% ============================================================================
\begin{slide}{SLIDE 5 -- Objectif de l'Application}
\textit{Durée : 45 secondes}
\end{slide}

\begin{texte}
\textbf{Ce que tu dis :}

\og Maintenant, pourquoi avons-nous développé cette application ?

Le problème, c'est que calculer les xVA à la main est \textbf{impossible}. Il faut :
\begin{itemize}
    \item Simuler des \textbf{milliers de scénarios} futurs de marché
    \item Calculer l'exposition du portefeuille à \textbf{chaque date} future
    \item Appliquer le \textbf{collatéral} pour réduire l'exposition
    \item Et intégrer les \textbf{probabilités de défaut} de la contrepartie
\end{itemize}

Notre solution, c'est cette application.

Le workflow est simple, comme vous le voyez sur le schéma :
\begin{enumerate}
    \item On définit un \textbf{portefeuille} de produits dérivés (des swaps de taux, des forwards de change)
    \item L'application fait une simulation \textbf{Monte Carlo} avec 5000 scénarios
    \item Et elle produit en sortie tous les \textbf{xVA} ainsi que le capital réglementaire \textbf{SA-CCR}
\end{enumerate}
\fg{}
\end{texte}

\begin{action}{Passe à la slide suivante}
\end{action}

\newpage

% ============================================================================
% SLIDE 6 : TECHNOLOGIES
% ============================================================================
\begin{slide}{SLIDE 6 -- Technologies \& Architecture}
\textit{Durée : 45 secondes}
\end{slide}

\begin{texte}
\textbf{Ce que tu dis :}

\og Quelques mots sur les technologies utilisées.

L'application est développée en \textbf{Python}, avec les librairies classiques de calcul scientifique : \textbf{NumPy} et \textbf{Pandas}.

Pour l'interface web interactive, nous utilisons \textbf{Streamlit}, qui permet de créer rapidement des dashboards.

Les graphiques sont générés avec \textbf{Plotly} pour avoir des visualisations dynamiques.

En termes de modèles financiers, nous utilisons :
\begin{itemize}
    \item Le modèle \textbf{Ornstein-Uhlenbeck} pour simuler les taux d'intérêt
    \item Le modèle \textbf{GBM}, Geometric Brownian Motion, pour le taux de change
\end{itemize}

À droite, vous voyez l'architecture du code. Le projet est organisé en modules :
\begin{itemize}
    \item \texttt{market} pour les modèles de marché
    \item \texttt{instruments} pour les produits financiers
    \item \texttt{exposure} pour le calcul d'exposition
    \item \texttt{collateral} pour la gestion du collatéral
    \item \texttt{xva} pour le calcul des ajustements
    \item Et \texttt{reg} pour le capital réglementaire SA-CCR
\end{itemize}

Mes collègues présenteront chacun de ces modules en détail. \fg{}
\end{texte}

\begin{action}{Passe à la slide suivante}
\end{action}

\newpage

% ============================================================================
% SLIDE 7 : LES 9 ONGLETS
% ============================================================================
\begin{slide}{SLIDE 7 -- Interface : Les 9 Onglets}
\textit{Durée : 45 secondes}
\end{slide}

\begin{texte}
\textbf{Ce que tu dis :}

\og Voici la structure de l'interface de notre application. Elle comporte \textbf{9 onglets}.

\begin{itemize}
    \item \textbf{Portfolio} : c'est là qu'on définit les trades du portefeuille
    \item \textbf{Exposure} : affiche les profils d'exposition EPE, ENE et PFE
    \item \textbf{xVA} : donne le détail de chaque ajustement de valorisation
    \item \textbf{SA-CCR} : calcule le capital réglementaire selon la méthode Bâle III
    \item \textbf{Calibration} : permet de calibrer les paramètres des modèles
    \item \textbf{Stress Test} : pour tester des scénarios extrêmes
    \item \textbf{Sensitivités} : calcule les sensibilités des xVA aux paramètres
    \item \textbf{Méthodologie} : documente toutes les formules utilisées
    \item Et \textbf{Export} : pour télécharger les résultats en Excel, CSV ou JSON
\end{itemize}

Le workflow typique est : définir le portefeuille, lancer la simulation, analyser l'exposition, regarder les xVA, vérifier le SA-CCR, et exporter les résultats. \fg{}
\end{texte}

\begin{action}{Passe à la slide suivante}
\end{action}

\newpage

% ============================================================================
% SLIDE 8 : PORTEFEUILLE
% ============================================================================
\begin{slide}{SLIDE 8 -- Portefeuille de Démonstration}
\textit{Durée : 30 secondes}
\end{slide}

\begin{texte}
\textbf{Ce que tu dis :}

\og Pour la démonstration, nous allons utiliser un portefeuille de test composé de \textbf{5 trades}.

\textbf{3 swaps de taux d'intérêt} :
\begin{itemize}
    \item Le premier : 10 millions de dollars, taux fixe 2.5\%, maturité 5 ans, on paie le fixe
    \item Le deuxième : 15 millions, taux 2\%, 3 ans, on reçoit le fixe
    \item Le troisième : 8 millions, taux 3\%, 7 ans, on paie le fixe
\end{itemize}

Et \textbf{2 forwards de change} EUR/USD :
\begin{itemize}
    \item Le premier : 5 millions d'euros, strike 1.12, maturité 1 an, on achète l'euro
    \item Le deuxième : 3 millions d'euros, strike 1.08, 2 ans, on vend l'euro
\end{itemize}

Au total, le notionnel est d'environ \textbf{44 millions de dollars}.

Passons maintenant à la démonstration live de l'application. \fg{}
\end{texte}

\begin{action}{Passe à la slide suivante (slide de démo), puis bascule vers l'application Streamlit}
\end{action}

\newpage

% ============================================================================
% DÉMO LIVE
% ============================================================================
\begin{slide}{SLIDE 9 -- Démonstration Live (+ Application)}
\textit{Durée : 3-4 minutes}
\end{slide}

\begin{action}{Bascule vers l'application Streamlit (Alt+Tab ou clique sur la fenêtre)}
\end{action}

\vspace{0.5cm}

\begin{texte}
\textbf{Ce que tu dis en montrant l'application :}

\og Je vais maintenant vous faire une démonstration de l'application.

\vspace{0.3cm}

\textbf{[MONTRE LE SIDEBAR À GAUCHE]}

À gauche, vous voyez le panneau de configuration. On peut modifier tous les paramètres :

\begin{itemize}
    \item Dans \textbf{Monte Carlo} : le nombre de simulations, ici 5000 paths, l'horizon de 5 ans, et le pas de temps trimestriel

    \item Dans \textbf{Market Models} : les paramètres des modèles de taux. Kappa c'est la vitesse de retour à la moyenne, theta c'est le taux long terme à 2\%, et sigma c'est la volatilité à 100 points de base

    \item Dans \textbf{Correlations} : les corrélations entre les différents facteurs de risque

    \item Dans \textbf{Collateral} : le seuil d'appel de marge à 1 million, le montant minimum de transfert à 100 mille, et la période de risque de 10 jours

    \item Et dans \textbf{Credit \& Funding} : le LGD à 60\%, les taux de défaut, le spread de funding et le coût du capital
\end{itemize}

\fg{}
\end{texte}

\newpage

\begin{texte}
\textbf{[MONTRE L'ONGLET PORTFOLIO]}

\og Dans l'onglet \textbf{Portfolio}, on voit les trades que j'ai décrits tout à l'heure. On a nos 3 swaps de taux et nos 2 forwards de change. On peut ajouter ou supprimer des trades directement dans cette interface.

\vspace{0.5cm}

\textbf{[CLIQUE SUR LE BOUTON RUN]}

Je vais maintenant lancer la simulation en cliquant sur le bouton \textbf{Run}.

\textit{[Attends quelques secondes que le calcul se fasse]}

La simulation est terminée. Elle a généré 5000 scénarios sur 5 ans. \fg{}
\end{texte}

\vspace{0.5cm}

\begin{texte}
\textbf{[CLIQUE SUR L'ONGLET EXPOSURE]}

\og Allons voir les résultats dans l'onglet \textbf{Exposure}.

Ici vous voyez les \textbf{profils d'exposition} :

\begin{itemize}
    \item La courbe \textbf{EPE}, Expected Positive Exposure, en bleu. C'est notre exposition moyenne quand elle est positive. Elle monte puis redescend vers la fin car les trades arrivent à maturité.

    \item La courbe \textbf{PFE} au-dessus représente le pire cas à 95\%. C'est notre exposition maximale dans 95\% des scénarios.

    \item Et vous pouvez voir l'effet du \textbf{collatéral} : la courbe verte montre l'exposition AVEC collatéral. Elle est nettement plus basse, environ 50\% de réduction.
\end{itemize}

C'est exactement ce qu'on veut : le collatéral réduit notre risque de moitié. \fg{}
\end{texte}

\newpage

\begin{texte}
\textbf{[CLIQUE SUR L'ONGLET xVA]}

\og Maintenant, l'onglet le plus important : \textbf{xVA}.

Ici vous avez le \textbf{breakdown complet} de tous les ajustements de valorisation :

\begin{itemize}
    \item Le \textbf{CVA}, le coût du risque de défaut de la contrepartie. C'est généralement la composante la plus importante.

    \item Le \textbf{DVA}, qui est un bénéfice, donc il est négatif dans le total.

    \item Le \textbf{FVA}, le coût de financement.

    \item Le \textbf{MVA}, le coût de la marge initiale.

    \item Et le \textbf{KVA}, le coût du capital réglementaire.
\end{itemize}

Le \textbf{Total xVA} est affiché en bas. C'est le montant total qu'il faudrait charger au client pour couvrir tous ces risques.

Vous voyez aussi le résultat exprimé en \textbf{points de base} par rapport au notionnel, ce qui permet de comparer facilement avec d'autres trades. \fg{}
\end{texte}

\vspace{0.5cm}

\begin{texte}
\textbf{[CLIQUE SUR L'ONGLET SA-CCR]}

\og Enfin, l'onglet \textbf{SA-CCR} montre le calcul du capital réglementaire selon la méthode standardisée de Bâle III.

Vous voyez :
\begin{itemize}
    \item Le \textbf{Replacement Cost}, qui est la valeur de marché actuelle
    \item Le \textbf{PFE}, Potential Future Exposure, calculé avec les supervisory factors
    \item Et l'\textbf{EAD}, Exposure at Default, qui est égal à 1.4 fois la somme des deux
\end{itemize}

Le 1.4 est le multiplicateur réglementaire fixé par le Comité de Bâle.

C'est cet EAD qui est ensuite utilisé pour calculer le capital que la banque doit détenir. \fg{}
\end{texte}

\newpage

\begin{texte}
\textbf{[OPTIONNEL - SI TU AS LE TEMPS : ONGLET EXPORT]}

\og Pour finir, dans l'onglet \textbf{Export}, on peut télécharger tous les résultats :
\begin{itemize}
    \item En \textbf{Excel} pour une analyse détaillée
    \item En \textbf{CSV} pour les importer dans d'autres outils
    \item Ou en \textbf{JSON} pour l'intégration avec d'autres systèmes
\end{itemize}

\fg{}
\end{texte}

\vspace{0.5cm}

\begin{action}{Reviens aux slides (Alt+Tab)}
\end{action}

\vspace{0.5cm}

\begin{texte}
\textbf{[DE RETOUR SUR LES SLIDES]}

\og Voilà pour la démonstration de l'application.

Pour résumer, notre application permet de :
\begin{enumerate}
    \item Définir un portefeuille de produits dérivés
    \item Simuler l'évolution des marchés avec Monte Carlo
    \item Calculer les expositions avec et sans collatéral
    \item Obtenir tous les xVA : CVA, DVA, FVA, MVA, KVA
    \item Et calculer le capital réglementaire SA-CCR
\end{enumerate}

Je passe maintenant la parole à \textit{[prénom du collègue partie 2]} qui va vous expliquer les modèles de marché utilisés pour la simulation. \fg{}
\end{texte}

\newpage

% ============================================================================
% AIDE-MÉMOIRE
% ============================================================================
\section*{Aide-Mémoire : Termes Clés}

Si on te pose des questions pendant ta partie :

\vspace{0.5cm}

\begin{tabular}{|p{4cm}|p{10cm}|}
\hline
\textbf{Terme} & \textbf{Explication simple} \\
\hline
\textbf{xVA} & Ajustements de valorisation = coûts cachés des dérivés \\
\hline
\textbf{CVA} & Coût du risque que la contrepartie fasse faillite \\
\hline
\textbf{DVA} & "Bénéfice" si c'est nous qui faisons faillite (controversé) \\
\hline
\textbf{FVA} & Coût d'emprunter l'argent qu'on doit financer \\
\hline
\textbf{MVA} & Coût de bloquer de l'argent en garantie (marge) \\
\hline
\textbf{KVA} & Coût du capital que la banque doit garder (réglementation) \\
\hline
\textbf{Monte Carlo} & On simule 5000 futurs possibles pour faire des moyennes \\
\hline
\textbf{EPE} & Exposition Positive Espérée = combien on peut perdre en moyenne \\
\hline
\textbf{PFE} & Pire cas (95\% des scénarios sont en dessous) \\
\hline
\textbf{Collatéral} & Garantie en cash qu'on échange pour réduire le risque \\
\hline
\textbf{SA-CCR} & Méthode Bâle III pour calculer le capital réglementaire \\
\hline
\textbf{EAD} & Exposure at Default = ce qu'on utilise pour le capital \\
\hline
\textbf{Pourquoi 1.4 ?} & Multiplicateur de sécurité fixé par les régulateurs de Bâle \\
\hline
\textbf{Pourquoi 5000 paths ?} & Compromis entre précision et temps de calcul \\
\hline
\end{tabular}

\vspace{1cm}

\begin{conseil}
\textbf{Conseils pour la présentation :}
\begin{itemize}
    \item Parle lentement et clairement
    \item Regarde le public, pas seulement l'écran
    \item Si tu ne sais pas répondre à une question, dis : \og C'est une bonne question, mon collègue X qui présente la partie Y pourra mieux vous répondre. \fg{}
    \item Pour la démo, montre avec ta souris ce dont tu parles
\end{itemize}
\end{conseil}

\end{document}
